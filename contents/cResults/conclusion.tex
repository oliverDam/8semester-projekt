Based on the results in the experiment it was found that training the user with confidence score feedback compared to label feedback can not be linked to any significant improvement in performance evaluated through a Fitts' Law test. Furthermore, no significant improvement during a three day training period for either the control or the test group was detected. These findings are most likely due to the high index of difficulty, making it hard to draw any conclusions based on the Fitts' Law test. 

Contrarily, it appears that training the user with label feedback can lead to a closer clustering of EMG data compared to training with confidence score feedback. This can be assessed, as a significant improvement was found between the first and last dataset recorded for the control group. To further support this the EMG signal of the subjects who received label feedback clustered significantly closer than the test group on the last day of testing. This shows that training based on confidence scores might not be a way to improve performance, which should be examined further by the use of Fitts' Law tests with lower ID's and a higher number of training sessions. 