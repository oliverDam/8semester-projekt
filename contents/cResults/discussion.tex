The objective of the study was to investigate if exposing subjects to user training, in which confidence levels of movement class recognition was used as feedback, would show statistically significant improvement in performance in a classification-based myoelectric prosthetic control scheme, when compared to subjects who received single-class feedback. \\
The results showed no significant difference between the test and control group within the Fitts' Law test, in all comparisons between and within groups ($p > 0.05$). This meant that no group performed better compared to the other, and that neither of the groups managed to improve significantly during the three sessions of training and testing. The only significant difference ($p < 0.05$) between the groups were found in the training when performing the closed hand motion, where the test group performed worse than the control group. This difference could be the result of the training type, the number of subjects or a faster learning ability within the control group.\\
A main cause of the lacking development within the groups could be the result of higher ID's ($4.88~-~6.41$) compared to other studies ($1.59~-~3.46$) \cite{Scheme2013, Scheme2013a}. Several subjects struggled in reaching any targets, and if the subject was unable to reach any targets, all the Fitts' Law measures except CR were unusable in statistics. This resulted in a weaker statistical test, as fewer observations were included in the comparison of the remaining measures. In addition, this lead to problems when examining the results, as it was expected that the statistical differences would primarily be found when looking at other measures than CR, as they would offer better insight into the improvement of precision when completing the test.  \\
At the same time a high ID led to subjects becoming frustrated when they had troubles reaching targets. When observing the test it was clear that this frustration resulted in the subjects forgetting how to perform precise movements, which then led to further frustration. This factor could also have had an effect on the subjects performance. Significant improvement in development of movement precision might also take more than three sessions, and this could also be a cause of the lacking development of the subjects. In developing the understanding of precision there should also be a higher focus on rest, as this is a crucial part of the performance test. Some of the subjects did not understand the importance of returning to rest after a performed movement during user training, which might have been reflected in the performance test.\\
The above points should be taken into consideration in future studies when examining the use of confidence scores as visual feedback in user training to improve performance.

While examining the EMG data it was found that the within cluster distance between the centroid and the samples improved within the control group ($p < 0.05$) between the sessions. When applying a Tukey-Kramer correction it was found that the difference was between the first and third session ($p < 0.05$) where the mean distance improved from $502.02 \pm 274.88$ to $323.43 \pm 171.13$. This result shows that the control group became better at performing precise movements, as the EMG data was more closely clustered after training for the three sessions. \\
Furthermore, a significant difference ($p < 0.05$) was found when comparing the within cluster distance of the two groups of the third session, where the mean distance for the control group ($323.43 \pm 171.13$) was close to half of the distance within the test group ($584.34 \pm 250.02$). This lead to the assessment that the control group became better at performing the exact movements during data acquisition when compared to the test group.%, as there was no significant difference when comparing data in the other sessions. 

\subsubsection{Optimization of Study}
When implementing the performance test interface, the ID's and minimum number of DOF's used to reach a target, should be lowered in order for the subjects to reach a CR of 80\% to 100\%, as reported in previous studies \cite{Scheme2013, Scheme2013a}. This might yield a more clear indication of precision of the control, which is shown better in the other Fitts' Law measures. At the same time a lower ID would give the subjects a feeling of success rather than frustration when performing the test, which might encourage them to retain the interest and focus when carrying out the performance test.
A problem observed during the Fitts' Law test was that subjects were affected by the cursor being reset to origo. The subjects' current movement were carried over during the transitioning between targets. A suggestion to future studies is to include a transition break of 1 second when a new target appears. \\
During user training the subjects should be forced to get back to rest, in order to train the ability to dwell within a target in the performance test. This requirement was not implemented in the current training interface, but the importance of learning to rest when using classifiers should be examined in future studies.\\
In future testing, the number of sessions should be more than three. This was also found in \cite{Pan2017}, who similarly did not achieve significant improvement in performance following a short time user training intervention. In that relation it would be beneficial to examine the time it takes to improve performance in order to find the minimum number of sessions necessary to achieve higher precision when performing specific hand gestures. \\
%The three days of training and testing did not result in a significant performance, but it can be hypothesised that a week of testing might be sufficient to achieve a better control. 
At last a larger number of subjects could result in a better distribution within the groups, as some subjects were able to get close to 100 \% CR in the first or second session, while others struggled with reaching just one target during each session. 
