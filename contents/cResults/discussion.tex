The results showed no significant difference between the test and control group within the Fitts' Law test, with all comparisons between and within groups yielding p-values below 0.05. This means that none of the groups performed better than the other, and that neither of them managed to improve significantly during the three days of training and testing. The only significant difference ($p < 0.05$) between the groups were found in the training when performing the closed hand gesture, where the test group performed worse than the control group. This difference could be the result of either the training type or the number of subjects.

A main cause of the lacking development within the groups can be the result of a high ID compared to other studies. Several subjects had problems reaching any targets at all, and if the subject was unable to reach any targets, all the Fitts' Law measures except for CR could not be used in the statistical tests. This leads to problems when examining the results, as it was expected that the statistical differences would primarily be found when looking at other measures than CR, as they would offer better insight into the development of the precision when completing the test. 

At the same time a high ID led to subjects becoming frustrated when they had a hard time reaching targets. When overseeing the test it was clear that this frustration resulted in the subjects forgetting how to perform precise movements, which then again led to more frustration. This factor could also have had an effect on the subjects performance. Visible improvement in development of movement precision might also take more than three sessions, and this could also be a cause of the lacking development within the subjects. When developing the understanding of precision there should also be a higher focus on rest, as this is a crucial part of the target test. Some of the subjects did not understand the importance of getting back to rest in training, which might be reflected in the target test.

\subsection{Optimization of Study}
The above points should be taken into consideration when examining the use of uncertainty and confidence scores in training to improve performance further on. When building the system the ID should be adjusted so that in the test it is rather easy to get a CR of 80\% to 100\%, in order to focus on the precision of the control, which is shown better in the other Fitts' Law measures. At the same time a lower ID would give the subjects a feeling of success rather than frustration when performing the test, which might encourage them to retain the interest and focus when training and testing

Furthermore the target test should be developed in a way so that the position and movement from trying to reach a previous target can not affect the position when a new target appears in order to make the test equal for all subjects. At the same time the subjects should be forced to get back to rest in training in order to be able to stay still within a target in the test. This was not implemented in the current training interface, but the importance of learning to rest when using classifiers could be examined in further studies.

When doing further testing the number of sessions should be more than three, and a study to examine the time it takes to improve could be performed in order to find the minimum number of days it takes to achieve higher precision when performing specific hand gestures. The three days of training and testing did not result in a significant performance, but it can be hypothesised that a week of testing might be sufficient to achieve a better control. At last a higher number of test subjects could result in a better distribution within the groups, as some subjects were able to get close to 100 \% CR in the first or second try, while others struggled with reaching just one target. 

\subsection{Other Findings}
While examining the EMG data it was found that the within cluster distance between the centroid and the samples improved within the control group ($p < 0.05$) between the sessions. When applying a Tukey-Kramer correction it was found that the difference was between the first and third session ($p < 0.05$) where the mean distance improved from $502.02 \pm 274.88$ to $323.43 \pm 171.13$. This result shows that the control group became better at performing precise movements, as the EMG data was more closely clustered after training for the three sessions. 

Furthermore a significant difference ($p < 0.05$) was found when comparing the within cluster distance of the two groups, where the mean distance for the control group ($323.43 \pm 171.13$) was close to half of the distance within the test group ($584.34 \pm 250.02$). This could lead to the conclusion that the control group became better at performing the exact movements during data acquisition when compared to the test group, as there was no significant difference when comparing data in the other sessions. 

