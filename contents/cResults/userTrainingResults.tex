
%3 user training:  completion rate for total (ding's) contraction level and movement classification
\section{User Training} \label{sec:R:userTraining}
This section covers the results acquired from measurements obtained during user training sessions. During user training subjects were instructed to train movements in being performed such that the control system recognized the movement as the actually performed movement. During this training the number of times subjects correctly performed an instructed movement to the contraction interval shown in the training interface was recorded, and will be referred to as the number of repetitions.
%This section will present the results acquired from measurements obtained during user training sessions. During user training subjects were instructed to train the performance of the six chosen movements. During this training it were recorded the number of times subjects correctly performed a instructed movement to the contraction interval shown in the training interface.% Statistical test were run on this data in a similar fashion as with the Fitts' Law results, described in \secref{sec:R:fitts}.  

\subsection{Total Completion Rate}
No significant difference in the total number of repetitions was found between sessions of either group ($p > 0.05$). When comparing the total number of repetitions of each session between groups accordingly, no significant difference were found either ($p > 0.05$).
%During user training the total completion rate is defined by the number of times a subject correctly performed a movement and held the contraction bar at the given interval until completion. The user training interface is described in \secref{sec:M:usertraining}. 
%
%A p-value of $p > 0.05$ was found for both groups in the Friedmans test comparing the performance over the three sessions. A significant difference was not found in any of the cases when performing Tukey-Kramer correction on the between-session Friedmans tests $p > 0.05$. This means that there was no significant development of performance in the training for any of the two groups. 

\subsection{Contraction Levels}
An increased ability to reach the low intensities was found for the control group ($p < 0.05$, session 1 $ = 16.13 \pm 5.59$, session 3  $ = 21.38 \pm 6.78$). Otherwise, similar results were yielded for both groups when comparing the subjects' ability to reach the three other contraction levels between sessions ($p > 0.05$).\\ No difference was found, when comparing the two groups' ability to reach different intensities during training, either ($p > 0.05$).
%Friedmans test was applied to examine if there was a development in the ability to reach the different contraction levels within the three training sessions. A p-value of $p > 0.05$ was found for both groups, with the Tukey-Kramer correction yielding $p > 0.05$ for the comparison of the three sessions. This means there was no significant development in the ability to reach different intensities within the user training. No difference was found between the two groups ability to reach different intensities during training ($p > 0.05$).

\subsection{Ability to Perform Movements}
Comparing the ability to perform different movements during the training showed a significant improvement for the test group in ulnar deviation ($p < 0.05$, session 1 $ = 11.38 \pm 4.27$, session 3 $ = 16.13 \pm 2.95$) and open hand ($p < 0.05$, session 1 $ = 11.25 \pm 3.85$, session 3 $ = 17.88 \pm 2.46$). A significant decrease in performance was found for the control group's ability to perform flexion ($p < 0.05$, session 2$ = 16.63 \pm 2.77$, session 3$ = 11.00 \pm 3.16$). Otherwise no significant difference between the three sessions for the two groups was found ($p > 0.05$). 
A significant difference ($p < 0.05$) was found between the test and control groups ability to reach the closed hand movement, with a mean of $26.8 \pm13.5$ number of repetitions for the test group and $38 \pm12.2$ for the control group. No significant difference was found for any of the other movements when comparing the two groups ($p > 0.05$).
%Comparing the ability to reach different positions within the training showed no significant difference between the three sessions ($p > 0.05$), with the Tukey-Kramer correction resulting in $p > 0.05$ between all sessions for both the test and control group. This shows that there was no significant development in the ability to reach different positions during training. 
%
%A significant difference ($p < 0.05$) was found between the test and control groups ability to reach the closed hand movement, with a mean of $26.8 \pm13.5$ for the test group and $38 \pm12.2$ for the control group. No significant difference was found for any of the other movements when comparing the two groups ($p > 0.05$).