%Powell plot example til data separability
\section{Data acquisition/separability results}
In this section results from the data acquisition are presented. The data used for training the LDA based classifier was examined. Each movement resulted in a cluster of data points, which are examined in this section, in order to analyse the change in cluster dispersion and distance between clusters centroids.
%In this section results from the data acquisition will be presented. The data from the data acquisition was used for training LDA classifier. This data were classified into regions decided by the LDA classifier used as control scheme in this project. For each region there was a class consisting of a cluster of data points. In this section the results of the analysis of the clusters are presented. %by use of PCA, described in \secref{sec:BG:dataSeparability}. With PCA the separability and density of data clusters can be evaluated.


\subsection{Between cluster distances}
For both groups the mean distance between the cluster centroids were calculated. The change in between cluster distances over the three sessions showed no significant difference for both groups ($p > 0.05$). Likewise, no significant difference of cluster distances between the groups was found ($p > 0.05$).\\
%For both groups the mean distance between the cluster centroids were calculated. There was found no significant difference in the development of cluster distances between the groups ($p > 0.05$). Likewise, the between cluster distances were tested between sessions, where no significance were found with a p-value of $p > 0.05$.


\subsection{Within cluster distances}
The mean distance from data points to the cluster centroid was calculated. This showed no significant difference for the test group ($p > 0.05$), but a significant difference was found for the control group ($p < 0.05$). The Tukey-Kramer correction showed the significant difference was between session one and three ($p < 0.05$), where the mean for session one was $502.02 \pm 274.88$, and session three was $323.43 \pm 171.13$. The comparison between groups showed that the control group achieved a significant improvement of within cluster distances compared to the test group in session three ($p < 0.05$), where the test group had a mean distance within clusters of $584.34 \pm 250.02$, while the control group had $323.43 \pm 171.13$.
%The mean distance from data points of a cluster to the cluster centroid were also calculated. Here there was found no significant difference for the test group ($p > 0.05$), but for the control group a significant difference with a p-value of $p < 0.05$ were found. Here the Tukey-Kramer correction showed that the significant difference were found between the control group's session one and three ($p < 0.05$), where the mean distance within clusters for session one were $502.02 \pm 274.88$, and for session three were $323.43 \pm 171.13$. 
%
%Results also show that the control group achieved a significant improvement when compared to the test group, in the mean distances within clusters ($p < 0.05$). In the third session the test group had a mean distance within clusters of $584.34 \pm 250.02$, while the control group had $323.43 \pm 171.13$. 


