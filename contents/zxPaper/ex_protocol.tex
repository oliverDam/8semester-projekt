\subsection{Experiment protocol}
Each subject underwent three sessions; one session per day over three consecutive days. The subjects were randomly allocated\fxnote{Except Christian Korfitz} in either a test or control group; 8 subjects in each group. During each session EMG-signals were initially acquired from the subjects and used to train the control system. The subjects then underwent user training with the purpose of learning how to adapt to the control system. Finally the subjects went through an online performance test to evaluate their ability to operate a virtual prosthesis. In the first session the subject did the performance test after data acquisition, which was used as a baseline assessment of the subject's performance. \\
The difference between the test and control groups, and the main area of interest in the study, lied in the feedback provided during user training. The test group was given visual feedback on exact confidence scores for each movement class when performing a movement, where the control group only was informed visually on the most recognized movement. A flowchart of the study design can be seen in \figref{fig:P:std}


\begin{figure}[H]                                         
	\includegraphics[width=0.64\textwidth]{figures/Paper/Study_design}  
	\caption{Graphical illustration of the experiment showing the steps of each session for the test and control group. Highlighted is user training in step 3 which was the only procedure that varied between the two groups, and comprised the main area of research interest in the experiment.}
	\label{fig:P:std} 
\end{figure}