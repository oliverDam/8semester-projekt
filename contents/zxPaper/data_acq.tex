\subsection{Data acquisition}
The recording of EMG-signals was archived with the Myo armband(MYB) from Thalmic Labs - an eight channel dry stainless steel electrode armband. MYB samples at 200 Hz, has a built in 50 Hz notch filter and a Bluetooth 4.0. unit which enables wireless communication with a computer. A 2_{nd} order Butterworth high-pass filter with a 10 Hz cut-off was digitally implemented to reduce movement artefacts. Due to the low sampling with no beforehand low-pass filtering, the aliasing of the signal was inevitable, thus no anti-aliasing filter was implemented. Despite the low sampling rate, MYB has shown to provide EMG signal recordings that can be classified with significantly similar accuracy as EMG signal recordings acquired with conventional EMG surface electrodes sampled at 1000 Hz \cite{Mendez2017}. \\
The subjects were instructed to elicit muscle contractions corresponding to the following classes of hand movements: \textit{Wrist extension, Wrist flexion, Radial deviation, Ulnar deviation, Closed hand, Open hand and Rest}, which are illustrated in \figref{hand_movements}. Initially the subjects were instructed in disinfecting their dominant forearm, and wear the MYB at the thickest part. To ensure the same placement of the MYB on each subject, the main electrode-channel was placed most laterally when standing in the anatomical standard position. The subjects were seated and instructed in relaxing shoulder and upper arm during the whole experiment. \\
According to Scheme et al. \cite{Scheme2015}, the use of dynamically changing contraction data in training a classification-based control scheme has shown to improve performance and tolerance to proportional control. Based on this finding, the subjects performed three repetitions of each movement, where each repetition constituted of a 3 second increasing ramp motion, a 5 second steady state motion at the peak of the increasing ramp motion and a 3 second decreasing ramp motion. To assure that each repetition was carried out correctly, the subjects were instructed to track a cursor, representing the EMG signal, on a trapezoidal trajectory, where the slopes corresponded to the ramp motions and the plateau corresponded to the steady state motion. The plateau of the trajectory differed between the three repetitions as 40 \%, 50 \% and 70 \% of an initial recorded 15 second constant force Maximum Voluntary Contraction (MVC). To avoid muscle fatigue the subjects were given 30 seconds rest after an MVC recording and 10 seconds rest between repetitions. 