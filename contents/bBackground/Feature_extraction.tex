\subsection{Feature extraction}

The raw EMG-signal itself is not used for myoelectric prosthesis control, but features that are extracted from it. % Thereby reducing the amount of redundant information limiting it to its most useful properties, resulting in faster computational speed. 
There are numerous feature components from an EMG signal which can be extracted either from the time-domain, frequency-domain, or time-frequency domain. Most used are features from the time- and frequency-domain. Time-domain features can be categorized in five different types based on their mathematical properties: energy information complexity information, frequency information, prediction modelling and time-dependency. Extracting features from the frequency-domain requires a frequency transformation, calculating the spectral properties of the recorded signal, which takes up longer processing time than simply using time-domain features. 
Time-domain features are often chosen based on their quick and easy implementation. They do not require any transformation before extraction and are calculated based on the raw EMG-signal. In addition, it is important not to choose redundant features for the classifier; features containing similar information. \cite{Phiny2012} 