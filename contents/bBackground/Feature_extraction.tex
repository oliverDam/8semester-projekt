
\subsection{Feature extraction}

Before using the recorded EMG-signal for any myoelectric prosthesis control, features are often extracted from the original signal and used for control instead. Thereby reducing the amount of redundant information limiting it to its most useful properties, resulting in faster computational speed. 
There are numerous feature components from an EMG signal which can be extracted either from the time-domain, frequency-domain, or as a time-frequency domain. Most used are features from the time- and frequency-domain. Extracting features to the frequency-domain requires some sort of frequency analysis, showing the spectral properties of the recorded signal, which takes up longer processing time than simply using the direct time--domain. 
Time-domain features are often chosen base on their quick and easy implementation. They do not require any transformation before extraction and are calculated based on the raw EMG-signal.\cite{Phiny2012} 