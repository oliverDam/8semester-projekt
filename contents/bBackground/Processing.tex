\section{Data processing} \label{sec:BG:dataProcessing}


In order to use the acquired EMG-signal in myoelectric prosthesis control, it first has to be processed, this is refereed to as pre-processing. Since the acquisition and most processing is done in the MYB before Bluetooth transmission, further processing of the signal is moderate. In myoelectric prosthesis control features are extracted for use in control, instead of using the entire EMG-signal. Hereby the amount of information is reduced resulting in faster computational speed. The following two sections will briefly describe theory behind filtering and feature extraction in relation to this project.  


\subsection{Filtering} \label{sub:BG:filtering} %\label{sub:filt}

Filtering is a cornerstone in preparing an EMG-signal for any kind of use. The frequency spectrum of EMG is 10 Hz to 500 Hz and most electrodes has a working range of 0 Hz to 500 Hz \cite{DeLuca2010}. According to the Nyquist theorem, to achieve a loss-less representation of the signal the sampling frequency must be at least twice the maximum frequency of interest of the original signal \cite{Pozzo2004}. Besides sampling with twice the maximum frequency, EMG is sensitive to artifacts of movement and electrical interference. Due to these circumstances, filters are often implemented to remove these unwanted contributors \cite{DeLuca2010}. 
General practice in filtering the EMG-signal will include implementing a notch filter with very narrow width and steep slope, at frequencies 49-51 Hz or 59-61 Hz depending on the power supply. The intend is to remove any electrical interference noise. In the low frequency spectrum several recommendations (5 Hz, 10 Hz and 20 Hz) has been made for optimal corner frequency of a high pass filter, to remove noise. A low pass filter is also typically used to remove any noise and unwanted signal above 500 Hz \cite{Cram2012}. 

This project will utilize a MYB for data acquisition and as mentioned in \subref{sub:BG:MYB} the MYB has a sample rate of 200 Hz. In relation to this project a sampling frequency of at least twice the maximum of the recorded signal is not possible, since muscles of the forearm have a maximum frequency of 400-500 Hz \cite{Cram2012}. This would require a sample rate of at least 1000 Hz, which cannot be achieved due to limitations in the MYB. Under other circumstances it would be astute to implement an anti-aliasing filter, this however is not possible with the MYB since an anti-aliasing filter should be implemented before the sampling. As this happens inside the MYB as fabricated by Thalmic Labs it is impossible to change for this projects. 

%The effect of the low sample rate of the MYB is aliasing in the recording, causing a frequency component not originally in the EMG signal. To account for this it would be resourceful to implement a low pass filter to act as an anti-aliasing filter.  

\subsection{Feature extraction} \label{sub:BG:featureExtraction} % \label{sub:BG:featureExtraction} %\label{sub:fe}

The raw EMG-signal is not itself used for myoelectric prosthesis control, but features that are extracted from it. Thereby reducing the amount of information limiting it to its most useful properties, resulting in faster computational speed. \\
There are numerous feature components from an EMG signal which can be extracted either from the time-domain, frequency-domain, or time-frequency domain. Most used are features from the time- and frequency-domain. Time-domain features can be categorized in five different types based on their mathematical properties: energy information, complexity information, frequency information, prediction modelling and time-dependency. Extracting features from the frequency-domain requires a frequency transformation, calculating the spectral properties of the recorded signal, which takes up longer processing time than simply using time-domain features. 
Time-domain features are often chosen based on their quick and easy implementation as they do not require any transformation before extraction and are calculated based on the raw EMG-signal. In addition, it is important not to choose redundant features for the classifier which would be to chose features providing similar information. \cite{Phiny2012} 

Extracting features for real-time prosthesis control is done by taking segments of the continuous signal, called windows. Calculation on extracting features are done in these discreet windows. This is done instead of using the instantaneous value due to the signals random nature. These windows are often overlapped to create a dense information stream for extraction. The relationship between window and overlap length is significant, when trying to determine the best representation. The window length is a matter of getting enough samples to do the calculation, but too long a window will result in delays slowing the control. Overlapping the window is a way to faster acquire windows by reusing a determined last segment of the prior window. Smith et al. \cite{Smith2014} found that the optimal window length in a classification control scheme that enables best performance ranges from 150-250 ms. 
