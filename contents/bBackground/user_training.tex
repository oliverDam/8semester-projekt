
\section{User training}

As user training in relation to prosthetic control is the main focus of this project an understanding of this concept in relation to receiving a prosthetic device is of great importance. Therefore the following section will cover an introduction to the concept of user training and its importance when preparing a subject to receive a prosthetic device. In addition some of the prior techniques of conducting user training will be presented, facilitating the possibility of assembling a user training protocol based on the most recent and cutting edge results.   

%user training and implementation
When fitting an amputee with a prosthesis, the way the prosthesis is controlled is important. A lot of work lies both ahead and behind fitting a person with a prosthesis. When developing and manufacturing a prosthesis two concepts emerge, one being system training and the other being user training. System training is training the control system to be able to recognize and differ movements based on the EMG-signal being feed to the system. \cite{Fougner2012} User training on the other hand focuses on training the user in performing distinguishable movements which can recognized by the control system. Here different types of feedback can be used to inform the user on how well it performs movement or how well the system recognizes the users performed movements. \cite{Powell2014,Simon2013}


%Copied from introduction 
Only few studies have earlier explored the optimal way of giving visual feedback in user training \cite{Jiang2012}
In a 2014 study Powell et al. \cite{Powell2014} provided the user with real-time visual feedback of a virtual prosthetic. This type of feedback is similar to the visual feedback a prosthesis user would receive using a normal prosthesis, although without the sensory feedback of the weight of the prosthesis. Pan et al. \cite{Pan2017} provided a visual feedback of an arrow to be moved on a 2D plane. The arrow was controlled by two DOF's; one controlled the horizontal position of the arrow, while the other could rotate the arrow \cite{Pan2017}. Fang et al. \cite{Fang2017} provided real-time visual feedback of subjects performed movement in relation to the classes defined in the system. The feedback visualized a map of clusters of different classes which subjects could match the position of a cursor to. When subjects could match the cursor to the centroid of a cluster the performed movement corresponded the best with the class of that movement. \cite{Fang2017} All studies observed an improvement in user performance after being exposed to focused user training with visual feedback. 
