



As presented earlier in \fxnote{ref to emg} the source of sEMG signal is the motor unit action potentials. The energy generated in action potentials is of a very small size and is measured in microvolts. Very sensitive recording equipments is therefore key in doing electromyography. Essential is to consider the type of electrode intended to use. Electrodes come in varies different sizes and shapes and are therefore very depended on the intended measurement site. Typically are electrodes made of silver-impregnated plastic used. They present desired characteristics by being disposable, relatively low price and by having low impedance with the skin. Most electrodes are covered with some adhesive compound in order form them to stick to the skin. These can either be 'dry' or covered with different types of gel, in order to reduce impedance and thereby noise, getting a more accurate EMG recording. Dry electrodes do not use gel, but instead rely on the skin to sweat thereby decreasing the skin impedance. Dry electrodes should prove better to patients with sensitive skin.\cite{Cram2012} Different skin conditions may also effect the electrode-skin impedance. People with makeup, scale or much hair increases the impedance, why the site should be shaved or rinsed with an alcoholic wipe.   




 


