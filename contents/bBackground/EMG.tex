

\section{Recording Electromyography}

This project will utilize the method of electromyography (EMG) to record the muscle activation of the lower arm in relation to the gestures presented in\fxnote{ref to anatomi}. To develop theoretical background knowledge, a short introduction of the essentials of the signal and the technique of recording it will be presented. 


Electromyography is the detection of muscle activity based on the amount of neurological or electrical stimulation. The amount of activity is found by measuring the electric potential, an action potential causing a muscle contraction. The process of planning and executing a voluntary movement starts at the motor cortex in the brain, and propagates through the spinal cord to the lower motor neuron. As seen in \figref{fig:motor} the path from alpha motor neuron through the axon to the motor endplates is what makes up a motor unit. The alpha motor neuron originates from the spinal cord along the axon to the muscle it controls. The axon branches out to multiple muscle fibers through motor endplates innervating the muscle fibers. The number of motor units innervating a muscle depend on its characteritics and the purpose is serves. Muscle control that demand high prescion have a higher innervation than muscles used for more powerful contractions. The recruitment of motor units is another way of controlling the force of a muscle contraction depending on the force needed. Like the recriument of motor units, the frequency of activation can be modulated for generating a specific amount of force. A higher activation frequency leads to a higher generated force, but this also makes the muscle more prone to fatigue.\cite{Cram2012,Martini2012}       

\begin{figure}[H]                                         
	\includegraphics[width=.4\textwidth]{figures/motor_unit}  
	\caption{The figure describes the neural pathway from the alpha motor neuron to the innervated muscle fibers, making up a motor unit. \cite{Konrad2005}}
	\label{fig:motor} 
\end{figure}  

The essentials to understanding the application EMG is the excitation of muscle cells. The excitability of the muscle fibers play a crucial role in the making of a muscle contraction. The mechanisms of a contraction can be understood through a series of events. First the muscle cell membrane is at a resting potential of -80 to -90 mV, due to an equilibrium of NA+ and k+ through the intracelluer and extracelluar side, maintained by an ion pump. The before mentioned alpha motor neuron reaches the motor endplates where a transmitter substance is released. The substance alters the membrane characteristics and allows a greater flow NA+ into the cell. This causes the a membrane depolarization, changing the membrane potential. If a threshold of approximately -55 mV to +30 is reached action potential is formed traveling in both directions of the muscle fiber, as seen on \figref{fig:motor}. The membrane potential is quickly restored with a great outflow of NA+, resulting in a repolarization. The generation of the motor unit action potential in the muscle via the shift from depolarization to repolarization is what with an EMG recording. The EMG recording represents the signal as a summation of motor unit action potentials over the muscle fiber membranes.\cite{Cram2012,Martini2012} 

Recording EMG can be either trough the mostly used surface EMG (SEMG) or by intra vascular EMG (IEMG). In IEMG a needle is inserted into the muscle measuring the MUAP directly on sight. The more used SEMG uses electrodes measuring the MUAP on the skin surface.\cite{Cram2012} 




 
 



