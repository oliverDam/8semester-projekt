%background regression 
\section{Linear regression methods} \label{sec:BG:linearRegressionMethods}
Classification can be used together with regression methods to provide a combination of the two control scheme. The output from the LDA classifier can be set to only decide on which movement is performed, and not
%The output from the LDA classifier only decides on which movement that is performed, and not 
at which contraction level the muscles used in the given movement are performing. Thus, the prosthesis can not perform any movement.  In statistics linear regression is often used to determine relations between variables. This notion can also be applied for myoelectric prosthetic control. While classification only provides an output on which class is recognized, a linear regression model provides a continuous output value based on the input value. If the regression model is fitted with information on different contraction levels for a given movement, control proportional to the contraction level will be achieved \cite{Hwang2017, Hahne2014, Bruun2017}. In the overall control scheme the classifier can then be used to decide which movement is performed, and a regression model can decide at which contraction level the movement is performed at. Similarly as with the classifier, regressors needs to be trained based on data acquired from the user, where the features extracted from the raw EMG signal is used as input. This procedure is described in the following section. 
%The use of linear regression is often used in statistics to determine relations between variables. Regressions methods also has its use in control schemes of myoelectric prosthetics \cite{Hwang2017, Bruun2017, Hahne2014}. The difference between classification and regression is that classification attempt to classify similar patterns in recordings, between previously acquired data and new data, while regression methods provide a continuous output value based on the input value \cite{Mendez2017, Hahne2014}. %\fxnote{sidste sætning er taget fra indtroduktionen men det stykke skal nok skrives om/væk alligevel for det omhandler classifikation VS regression. edit: er taget ud af introduktionen nu.}

Different models of linear regression exist to account for different uses. When utilizing regression methods it must be considered which type of input variables are used and what type of relation these variables might have. The appropriate regression must then be applied. Simple linear regression approximate a relation between one dependent variable $Y$ and one independent variable $X$ \cite{Zar2009}:

\begin{equation} \label{eq:simpleLinearRegression}
Y = \alpha + \beta X + \epsilon
\end{equation}

where $Y$ is the control output for the prosthesis, $X$ is the feature extracted from the EMG signal, $\beta$ is the regression coefficient in the sampled population, $\epsilon$ is the error, and $\alpha$ is the predicted value of $Y$ at $X = 0$.
This model can be expanded to estimate relations between one dependent variable and several independent variables. This is called multivariate regression and expands on the equation of simple linear regression, given in \eqref{eq:simpleLinearRegression} \cite{Zar2009}:

\begin{equation} \label{eq:multiLinearRegression}
\hat{Y} = \alpha + \beta_1 X_{1} + \beta_2 X_{2} + ... + \beta_i X_{i} + \epsilon_i
\end{equation} 

where $i$ in this project would correspond to the number of channels in the MYB \cite{Zar2009}. Since this regression model approximates the relation between several independent variables and one dependent variable, this model can be used as a control scheme in myoelectric prosthetics, and . Here the channel-recordings of muscle activity can be considered independent variables, and used to estimate one control output, which would be the dependent variable. \cite{Bruun2017}

