\section{Statistical analysis}
When evaluating the scores obtained in the performance test a statistical analysis is used. For this project a Friedman's test will be utilized due to a small sample population, and because it is assumed that the probability distribution of the performance scores is unknown \cite{Zar2009}. Friedman's test is commonly used to test if different treatments have similar or different effects in a subject population. In the case of this project it is used to test whether performance in prosthesis control is similar or different across sessions in a subject population and if performance scores differ or bear resemblance between two subject populations. In the following section, theory on how the Friedman's test is calculated will be presented.

\subsection{Friedman's test}
The aim for the Friedman's test is to calculate the Friedman's F value to compare it with its corresponding critical F value to finally decide if the null-hypothesis or an alternative hypothesis should be accepted. The null-hypothesis expresses a relationship between measurements (the means of measurements are equal) and the alternative hypothesis expresses no relationship between measurements (the means od measurements are unequal).
When testing a subject population of a given number of measurements the measurements must first be arranged in columns, where each column corresponds to a certain measurement, and each row corresponds to one subject; this row is also called a block. Each block must then be ranked separately, where the smallest number is ranked 1. If numbers in a row are equal they get the mean of the rank they would have received. The sum $R$ of each column is then calculated, and the number of measurements $k$ and number of subjects $n$ are used to calculate the Friedman's F value in the following equation \cite{Zar2009}:

\begin{equation}
	F = \Big[\frac{12}{n \cdot k \cdot (k+1)}\Big] \sum_{i=1}^{k} R_{i}^{2} - 3 \cdot n(k + 1)
\end{equation}

The critical value of F is then determined by looking in a table of critical values for Friedman's test using the values for $k$, $n$ and a significance level $\alpha$. $\alpha$ needs to be chosen when looking up the critical values, where a level of 0.05 typically is selected; it is not excessively high to seize too many type 1 errors (rejecting a true null hypothesis) and not excessively low to seize too many type 2 errors (retaining a false null hypothesis). The F value and critical F value are then compared in order to decide whether to retain or reject the null-hypothesis. If the calculated F value is larger than the critical F value the null-hypothesis is rejected and vice versa. \cite{Zar2009} \\
Another method of deciding on which hypothesis to be accepted is to evaluate the probability value (p-value), which can be returned by using statistical software. A significance level of 0.05 is again usually chosen. If the p-value is under 0.05 the null-hypothesis is rejected and vice versa. \cite{Zar2009}

When comparing multiple groups of measurements the incident of rejecting a true null-hypothesis increase, and thus ignoring type 1 errors between pairs of means in the measurement groups.  Several methods exist to counteract this multiple comparison problem. In this project the Bonferroni correction will be utilized for this purpose.

\subsection{Bonferroni correction}
A total number of $\frac{k(k-1)}{2}$ pairs can be coupled in multiple comparison testing, thus a total number of $\frac{k(k-1)}{2}$ hypotheses can be defined. The Bonferroni correction counteracts the incorrect rejection of a null-hypothesis by lowering the significance level for each tested hypothesis by a scale $m$, where $m$ is the total number of hypotheses. Thus, the new significance level tested for each individual hypothesis is $\frac{\alpha}{m}$. The Bonferroni correction then rejects the null-hypothesis when p-value $< \frac{\alpha}{m}$. 


%"A type I error is the rejection of a true null hypothesis (also known as a "false positive" finding), while a type II error is retaining a false null hypothesis (also known as a "false negative" finding)"

%"0.05 is an arbitrary significance level (or ‘alpha’); not too high that we’re making too many Type I errors (assuming an effect where there isn’t one(false positive)), but not too low that were making too many Type II errors (assuming there isn’t an effect where there is one(false negative))."
