%paper introduction

\section{Introduction}			%skal måske fjernes afhængigt af hvordan vi sætter main op

Electromyography (EMG) is the recording and utilization of muscle generated electric potentials, widely used in control of functional prosthetics. The electric potentials recorded from the muscles are action potentials generated by the activation of a muscle contraction. The contraction force a muscle produce is related to the intensity of an EMG recording. The recorded EMG signals are processed through several steps of amplification, filtering and feature extraction before they are used for control in a myoelectric prosthesis. \cite{Cram2012, Fougner2012} For the actual control of the prosthetics several different methods for control schemes exist.

In recent years the research area of myoelectric prosthetics has been dominated by classification methods for control schemes. Classification attempt to classify similar patterns in EMG signals, between previously acquired data and new data \cite{Mendez2017}. This approach for control have proved adequate when performing proportional and simultaneous movements in several degrees of freedom (DOF), but have lacked usability outside of clinical environments due to poor accuracy with change in limb position. This have resulted in a high rejection rate by users. 
%small section on how classification methods function?
However, recently regression methods have been gaining more interest as a control scheme for myoelectric prosthetics. Regression methods provide a continuous output value, contrary to classification which will provide a single class per movement \cite{Hahne2014}. Regression methods have shown promising results of robust control while performing both proportional and simultaneous movements, while also having proved to effectively combat the effect of limb position changes \cite{Hwang2017, Hanhe2014}. This shows potential for regression based control schemes to be more reliable when used for performing daily life tasks, outside clinical environments.
%small section on how regression methods function?

However, in studies the use of regression based control have been directly based off of the output of a trained regression model. This approach have shown robust control across different limb positions but lacks accurate control when performing delicate/precise movements or when only performing movement in one DOF. [cite til 7.semester projekt]. Many advancements have been made on system training to improve the systems and control schemes to best recognize the performed movements by the users. However, far fewer studies have investigated the effect of user training, improving the users ability to properly utilize the system. Here, an important consideration is that each user will have individual competences when initiating user training. Some might perform well form the beginning while others will show little to no success. \cite{Powell2013} Powell et. al \cite{Powell2013} conclude that in order for amputees to understand the significance of producing consistent and distinguishable muscle patterns, the need for user training is important. User training can help amputees to gain the skill of controlling pattern recognition based prosthetics and to later adopt the use of one such prosthesis \cite{Powell2013}.

The significance of user training is not doubted, but an effective way to properly provide feedback to the user have yet to be developed. 
Fang et. al \cite{Fang2017} evaluated the progress of the human learning ability in a pattern recognition based control scheme when providing classifier-feedback during user training. Here, a clustering-feedback method based on Principal Component Analysis (PCA) was used to provide users with real-time visual feedback, to guide users to correctly perform movements based on the recorded EMG signals. The visual feedback consisted of a map with dots representing centroids of classes. Through control based on an Linear Discriminant Analysis (LDA) classifier, users could match the control input to these centroids to best perform a movement to be classified correctly. The study showed great improvements for user training, and an ability to quicken the learning for amputees who are unfamiliar with EMG controlled prosthetic use. \cite{Fang2017}
Other studies have also showed promising results using an LDA classifier during user training. Powell et. al \cite{Powell2014} demonstrated an increase in movement completion percentage from 70.8\% to 99.0\%, a decrease in movement completion time from 1.47 to 1.13, as well as a significant improvement in classifier accuracy from 77.5\% to 94.4\%, for users undergoing user training for a two week period. This study provided feedback through a virtual prosthesis.


%then some clever arguments for why we do what we do want to do

Ever increasingly advanced myoelectric prosthetics and control systems are being developed, a critical bottleneck still exist: the ability to properly control the prosthetic. In relation to pattern recognition methods the overall challenge lies in the ability for the system to be able to recognizing the muscle patterns produced by the user. Thus system training has become exceedingly good at this task. However, the challenge for user training is for the user to be able to consistently produce distinguishable muscle pattern, and the better these muscle patterns the better the system will function. \cite{Powell2014} Therefore further research in user training could provide a vital leap towards more precise classification using current methods, as well as a faster user adaptation of myoelectric controlled prosthetics. 


The remainder of this paper will be structured in "x numbers" of sections. Section 2 will further describe the experimental setup, subject management and experimental protocol. Section 3 will describe the methods used to do something with the control off and the user training thing we do. Section 4 present the result, discussion and conclusion. 

