%paper introduction

%\section{Introduction} \label{sec:introduction}			%skal måske fjernes afhængigt af hvordan vi sætter main op


%problem: despite advanced prosthetics and control systems users reject prosthetics
%why is this a problem? we want users to use prosthetics
%types of advanced prosthetics and how they are trained and controlled
%
%what can be done to solve/improve on the advanced prosthetics: training of some sort
%what have been done in system training
%what have been done in user training
%feedback: diff types (don't end on the tDCS source)  
%we do user training


The loss of any part of a limb or limb as a whole is a great cost for any human. The hand is one of the most precious tools humans have and thus a loss of this would prove to be a great loss of functionality and independence. In an effort to restore some of that ability and autonomy, many patients are provided with prosthetics. \\
In recent years, prosthetics have become exceedingly good in performance, however, lack of functionality and discomfort of prosthetics are causing patients to reject the provided prosthesis. \cite{Reiber2010}. Commercial available prosthetics range from passive cosmetic prosthetics to functional low degree of freedom (DOF) cable-driven prosthetics and switch controlled myoelectric prosthetics. \\
In recent years several complex multi DOF prosthetic hands have been developed. Examples of this are the the Vincent hand by Vincent Systems, iLimb hands from Touch Bionics, the Bebionic hands from RSL Steeper and the Michelangelo hand from Otto Bock \cite{Belter2013}. Despite the efforts to advance and improve the functionality of prosthetics, a critical bottleneck still exist: the ability to properly control the prosthetic \cite{Hwang2017}. The general challenge for users is to be able to consistently produce distinguishable muscle patterns, for the prosthetic control system to recognize. \cite{Powell2014} \\
Most commercially available myoelectric controlled prosthetics rely on switch control which is a robust control scheme, but is slow and non-biologic in movement. In the research area of myoelectric prosthetics newer control schemes have been developed. %, but is only used in one commercially available prosthetics to date. 
These control schemes are classification and regression. Classification have been used for many years in research but is to date only used in one commercially available prosthetic. When using classification as a control scheme the classifier attempts to classify similar patterns in electromyography (EMG) signals between previously acquired data and new data \cite{Mendez2017}. The regression control scheme determine the output signal for a input based on a regression line. This provide a continuous output value contrary to classification which provide a single value. \cite{Hahne2014} \\
Both types of control schemes have become exceedingly good at correctly estimating muscle patterns. \cite{Hahne2014, Bruun2017, Englehart2003, Scheme2015} However, there still exist a challenge for the users to be able to consistently produce distinguishable muscle patterns \cite{Powell2014}. In resent years many advancements have been made in research on system training. System training is the training of the control system to recognize the input signals from the user \cite{Fougner2012}. This area focus on the design of the hardware and software side of the system in EMG prosthetics. Jiang et al. \cite{Jiang2012} argue that a change should be made in the focus of research on myoelectric prosthetics in relation to improving control. The awareness in the research area show a very single-minded approach to possible improvements of control, and thus mainly system training have been researched. Jiang et al. \cite{Jiang2012} discuss that the awareness of possible other practical implementation have been underestimated. One such implementation which have been addressed in only a few studies is user training \cite{Fang2017, Powell2014, Pan2017}. Contrary to system training, user training focus on the user's ability to control a prosthetic \cite{Fougner2012}. User training differ from regular use of a prosthetic in that the training is part of the initial period, where the system is being adjusted to the individual user. Here different types of feedback can be used to inform the user on how well it performs movement or how well the system recognizes the users performed movements. \cite{Powell2014,Simon2013} \\
In a 2014 study Powell et al. \cite{Powell2014} provided the user with real-time visual feedback of a virtual prosthetic. This type of feedback is similar to the visual feedback a prosthesis user would receive using a normal prosthesis, albeit without the sensory feedback of the weight of the prosthesis. Pan et al. \cite{Pan2017} provided a visual feedback of an arrow to be moved on a 2D plane. The arrow was controlled by two DOF's; one controlled the horizontal position of the arrow, while the other could rotate the arrow \cite{Pan2017}. Fang et al. \cite{Fang2017} provided real-time visual feedback of subjects performed movement in relation to the classes defined in the system. The feedback visualized a map of clusters of different classes which subjects could match the position of a cursor to. When subjects could match the cursor to the centroid of a cluster the performed movement corresponded the best with the class of that movement. \cite{Fang2017} All studies observed an improvement in user performance after being exposed to focused user training with visual feedback. 

Studies investigating the effect of user training shows promising results, however, as Jiang et al. \cite{Jiang2012} discuss the myoelectric prosthetic research area might have been too focused on system training in resent years and could overall benefit from an expansion of research interests to include previously underestimated implementations or completely new approaches. \\
A 2013 study by Scheme et al. \cite{Scheme2013} proposed a novel approach of utilizing confidence-based rejection to improve system training of myoelectric control. Here a classification control scheme was provided with confidence scores to assist in acceptance or rejection of the class output. The confidence scores were calculated from a modification of Bayes' theorem. Scheme et al. \cite{Scheme2013} showed a significant improvement in performance with the use of the rejection-capable system when compared to the normal classification scheme. A similar approach could be used in user training by providing the confidence scores of the classification to the user as a form of visual feedback. \\
Thus, this study propose a novel method of providing users with feedback containing confidence scores for different classes in a classification control scheme during user training. Contrary to current feedback methods in user training this approach could enable users to better understand how the classification works based on their performed movements. During user training this could improve the way users perform specific movements in order to enable the system to better recognize and classify movements correctly.






 
 
%----------OLD INTRODUCTION----------%
% 
%users could match the control input to these centroids to best perform a movement to be classified correctly.
%
%studies by Feng, Pan, and Powell with less information on results than before. 
%
%
%Fang et. al \cite{Fang2017} used a clustering-feedback method based on Principal Component Analysis (PCA) to provide subjects with visual feedback to guide them to corre
%
%
%
%Fang et. al \cite{Fang2017} evaluated the progress of the human learning ability in a pattern recognition based control scheme when providing classifier-feedback during user training. Here, a clustering-feedback method based on Principal Component Analysis (PCA) was used to provide users with real-time visual feedback, to guide users to correctly perform movements based on the recorded EMG signals. The visual feedback consisted of a map with dots representing centroids of classes. Through control based on an Linear Discriminant Analysis (LDA) classifier, users could match the control input to these centroids to best perform a movement to be classified correctly. The study showed great improvements in performance after user training, and an ability to quicken the learning for amputees who are unfamiliar with EMG controlled prosthetic use. \cite{Fang2017}
%Other studies have also showed promising results using an LDA classifier during user training. Powell et. al \cite{Powell2014} demonstrated an increase in movement completion percentage from 70.8\% to 99.0\%, a decrease in movement completion time from 1.47 to 1.13, as well as a significant improvement in classifier accuracy from 77.5\% to 94.4\%, for users undergoing user training for a two week period. This study provided feedback through a virtual animated prosthesis.
%Pan et. al \cite{Pan2017} provided a visual feedback of an arrow to be moved on a 2D plane. Pan et. al also tested the effect of stimulating the subjects brain with transcranial direct current stimulation (tDCS). The study concluded that tDCS together with user training provided significantly better results than user training alone. \cite{Pan2017}
%
%
%
%
% should   of previously underestimated practical implementations on improving control have 
%
%
%
%Many advancements have been made on system training to improve the systems and control schemes to best recognize the performed movements by the users. Jiang et. al \cite{Jiang2012} argue that a change should be made in the focus of research on myoelectric prosthetics in relation to improving control. 
%
%of focus in relation to improving control, in the myoelectric prosthetics research area should be made. Perhaps as a result of a too single-minded approach in the research community, compared to system training, far fewer studies have investigated the effect of user training. 
%
%
%In resent years immense advancements have been made and many new types and models of prosthetics have been invented \cite{Belter2013}. The improvements in the prosthetic research field have advanced prosthetics from low degree of freedom (DOF) cable-driven or switch controlled prosthetics to complex multi DOF prosthetics. . 
%
%Electrically controlled prosthetics function on the recordings of muscle signals, called electromyography (EMG). Based on muscle contractions a myoelectric prosthesis is controlled, similarly to a biologic hand. The recording of EMG signals are used for control but goes through a process of steps of amplification, filtering and feature extraction before they can be used as input to the control of the prosthesis. \cite{Cram2012, Fougner2012} For the actual control of the prosthetics a control scheme is implemented. 
%
%
%
%
%Electromyography (EMG) is the recording of muscle generated electric potentials, widely used in control of functional prosthetics. The electric potentials recorded from the muscles are action potentials generated before the inception of a muscle contraction. The contraction force a muscle produce is related to the intensity of an EMG recording. The recorded EMG signals are processed through steps of amplification, filtering and feature extraction before they are used as input in the control for a myoelectric prosthesis. \cite{Cram2012, Fougner2012} %For the actual control of the prosthetics several different methods for control schemes exist.
%
%%the best introduction including the great works of Jiang \cite{Jiang2012} to say that an overall problem with myoelectric prosthetics exist in the fact that not everything has yet been investigated. However this should be done in order to know everything. Everything should be researched. 
%
%Ever increasingly advanced myoelectric prosthetics and control systems are being developed. Despite the efforts a critical bottleneck still exist: the ability to properly control the advanced prosthetic \cite{Hwang2017}. In relation to pattern recognition methods the overall challenge lies in the ability for the system to be able to recognizing the muscle patterns produced by the user. Control systems have become exceedingly good at correctly estimating muscle patterns. However, there still exist a challenge for the users to be able to consistently produce distinguishable muscle patterns, and the better these muscle patterns the better the system will function. \cite{Powell2014}
%
%In recent years the research area of myoelectric prosthetics has been dominated by classification methods for control schemes. Classification attempts to classify similar patterns in EMG signals, between previously acquired data and new data \cite{Mendez2017}. Classification enables proportional control of trained movements in several degrees of freedom (DOF), but but only a single movement a time. The classification control scheme has lacked usability outside of clinical environments \cite{Scheme2010}, which has resulted in scarce commercial success \cite{Jiang2012}.
%%Recently regression methods have been gaining more interest as a control scheme for myoelectric prosthetics. Regression methods provide a continuous output value, contrary to classification which provides a single class per movement \cite{Hahne2014}. Regression methods have shown promising results of robust control while performing both proportional and simultaneous movements \cite{Hwang2017, Hahne2014}. This shows potential for regression based control schemes to be more reliable when used for performing daily life tasks outside clinical environments.However, the regression-based control approach lacks accurate control when performing delicate movements or when the user only desires to perform a single movement in one DOF. [cite til 7.semester projekt].
%
%Many advancements have been made on system training to improve the systems and control schemes to best recognize the performed movements by the users. Jiang et. al \cite{Jiang2012} determine that a change of focus in the myoelectric prosthetics research area should be made. Perhaps as a result of a too single-minded approach in the research community, compared to system training, far fewer studies have investigated the effect of user training. 
%Improving the users ability to properly utilize the system is the goal of user training. Here, an important consideration is that each user will have individual competences when initiating user training. Some might perform well from the beginning while others will show little to no success. \cite{Powell2013} Powell et. al \cite{Powell2013} conclude that in order for amputees to understand the significance of producing consistent and distinguishable muscle patterns, the need for user training is important. User training can help amputees to gain the skill of controlling pattern recognition based prosthetics and to later adopt the use of one such prosthesis \cite{Powell2013}.
%
%The significance of user training is not doubted, and several different approaches has been investigated. Fang et. al \cite{Fang2017} evaluated the progress of the human learning ability in a pattern recognition based control scheme when providing classifier-feedback during user training. Here, a clustering-feedback method based on Principal Component Analysis (PCA) was used to provide users with real-time visual feedback, to guide users to correctly perform movements based on the recorded EMG signals. The visual feedback consisted of a map with dots representing centroids of classes. Through control based on an Linear Discriminant Analysis (LDA) classifier, users could match the control input to these centroids to best perform a movement to be classified correctly. The study showed great improvements in performance after user training, and an ability to quicken the learning for amputees who are unfamiliar with EMG controlled prosthetic use. \cite{Fang2017}
%Other studies have also showed promising results using an LDA classifier during user training. Powell et. al \cite{Powell2014} demonstrated an increase in movement completion percentage from 70.8\% to 99.0\%, a decrease in movement completion time from 1.47 to 1.13, as well as a significant improvement in classifier accuracy from 77.5\% to 94.4\%, for users undergoing user training for a two week period. This study provided feedback through a virtual animated prosthesis.
%Pan et. al \cite{Pan2017} provided a visual feedback of an arrow to be moved on a 2D plane. Pan et. al also tested the effect of stimulating the subjects brain with transcranial direct current stimulation (tDCS). The study concluded that tDCS together with user training provided significantly better results than user training alone. \cite{Pan2017}
%
%The general challenge of user training is for the user to be able to consistently produce distinguishable muscle patterns. \cite{Powell2014} Therefore further research in user training could provide a vital leap towards more precise classification using current methods, as well as a faster user adaptation of myoelectric controlled prosthetics, but an effective way to properly provide feedback to the user have yet to be developed. 
%%Further studies should for now concentrate on developing different feedback methods which should later be compared to determine an ideal method. 
%This study will seek to develop a new method of feedback during user training, by providing the users with the confidence scores of a LDA classifier as used in \cite{Scheme2013} for a confidence-based rejection system control scheme, and the level of contraction of the performed movement as proposed in \cite{Englehart2003}. To the authors knowledge, these feedback parameters have not yet been provided in a combination in user training for myoelectric prosthesis users. %Initially the study will propose providing feedback via a bar chart each bar representing the estimated uncertainty of a classified movement. 
%
%%The following section ought to be a overview section in the beginning of the method section:
%%The study will consist of steps of data acquisition, user training and an online test. The user will undergo user training followed by a targets reaching Fitts' Law test. The study will have a test and control group, where the control group will be trained without visual feedback during user training. Ad hoc statistic comparisons will be made between online tests and maybe some other things we have not entirely decided upon yet. /cite(we decide)
%
%The remainder of this paper will be structured in "x numbers" of sections. Section 2 will further describe the experimental setup, subject management and experimental protocol. Section 3 will describe the methods used to do something with the control off and the user training thing we do. Section 4 present the result, discussion and conclusion. 
%
