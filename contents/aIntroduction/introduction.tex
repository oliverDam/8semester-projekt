%paper introduction

\section{Introduction}			%skal måske fjernes afhængigt af hvordan vi sætter main op


In recent years the research area of myoelectric prosthetics has been dominated by pattern recognition methods for control schemes. However recently regression methods have been introduced and have shown promising results of robust control while performing both proportional and simultaneous movements. Furthermore in 2017, Bruun et al. showed that the use of regression methods could effectively combat the effect of limb position changes [kilde 1 og 2]. However the use of regression based control have been directly based off of the output af a trained regression model. This approach have shown robust control across different limb positions but lacks accurate control when performing delicate/precise movements or when only performing movement in one DOF. A possible way to increase accurate control is to include a probability/uncertainty estimation of movements. This can be achieved by combining regression and classification methods into one control scheme. (linear regression and LDA)

This study investigate the effect of implementing uncertainty estimation and probability of desired movements, when using linear regression as control scheme. 


\textbf{Keywords: platypus, onion, shoemaker, pepperspray, cold-ass-feet, sabre, duck.}