%paper introduction

\section{Introduction}			%skal måske fjernes afhængigt af hvordan vi sætter main op

%    https://www.youtube.com/watch?v=BzVmPsqHDDQ
Electromyography (EMG) is the recording of muscle generated electric potentials, widely used in control of functional prosthetics. The electric potentials recorded from the muscles are action potentials generated before the inception of a muscle contraction. The contraction force a muscle produce is related to the intensity of an EMG recording. The recorded EMG signals are processed through steps of amplification, filtering and feature extraction before they are used as input in the control for a myoelectric prosthesis. \cite{Cram2012, Fougner2012} %For the actual control of the prosthetics several different methods for control schemes exist.

%the best introduction including the great works of Jiang \cite{Jiang2012} to say that an overall problem with myoelectric prosthetics exist in the fact that not everything has yet been investigated. However this should be done in order to know everything. Everything should be researched. 

Ever increasingly advanced myoelectric prosthetics and control systems are being developed. Despite the efforts a critical bottleneck still exist: the ability to properly control the advanced prosthetic \cite{Hwang2017}. In relation to pattern recognition methods the overall challenge lies in the ability for the system to be able to recognizing the muscle patterns produced by the user. Control systems have become exceedingly good at correctly estimating muscle patterns. However, there still exist a challenge for the users to be able to consistently produce distinguishable muscle patterns, and the better these muscle patterns the better the system will function. \cite{Powell2014}

In recent years the research area of myoelectric prosthetics has been dominated by classification methods for control schemes. Classification attempts to classify similar patterns in EMG signals, between previously acquired data and new data \cite{Mendez2017}. Classification enables proportional control of trained movements in several degrees of freedom (DOF), but but only a single movement a time. The classification control scheme has lacked usability outside of clinical environments \cite{Scheme2010}, which has resulted in scarce commercial success \cite{Jiang2012}.
%Recently regression methods have been gaining more interest as a control scheme for myoelectric prosthetics. Regression methods provide a continuous output value, contrary to classification which provides a single class per movement \cite{Hahne2014}. Regression methods have shown promising results of robust control while performing both proportional and simultaneous movements \cite{Hwang2017, Hahne2014}. This shows potential for regression based control schemes to be more reliable when used for performing daily life tasks outside clinical environments.However, the regression-based control approach lacks accurate control when performing delicate movements or when the user only desires to perform a single movement in one DOF. [cite til 7.semester projekt].

Many advancements have been made on system training to improve the systems and control schemes to best recognize the performed movements by the users. Jiang et. al \cite{Jiang2012} determine that a change of focus in the myoelectric prosthetics research area should be made. Perhaps as a result of a too single-minded approach in the research community, compared to system training, far fewer studies have investigated the effect of user training. 
Improving the users ability to properly utilize the system is the goal of user training. Here, an important consideration is that each user will have individual competences when initiating user training. Some might perform well from the beginning while others will show little to no success. \cite{Powell2013} Powell et. al \cite{Powell2013} conclude that in order for amputees to understand the significance of producing consistent and distinguishable muscle patterns, the need for user training is important. User training can help amputees to gain the skill of controlling pattern recognition based prosthetics and to later adopt the use of one such prosthesis \cite{Powell2013}.

The significance of user training is not doubted, and several different approaches has been investigated. Fang et. al \cite{Fang2017} evaluated the progress of the human learning ability in a pattern recognition based control scheme when providing classifier-feedback during user training. Here, a clustering-feedback method based on Principal Component Analysis (PCA) was used to provide users with real-time visual feedback, to guide users to correctly perform movements based on the recorded EMG signals. The visual feedback consisted of a map with dots representing centroids of classes. Through control based on an Linear Discriminant Analysis (LDA) classifier, users could match the control input to these centroids to best perform a movement to be classified correctly. The study showed great improvements in performance after user training, and an ability to quicken the learning for amputees who are unfamiliar with EMG controlled prosthetic use. \cite{Fang2017}
Other studies have also showed promising results using an LDA classifier during user training. Powell et. al \cite{Powell2014} demonstrated an increase in movement completion percentage from 70.8\% to 99.0\%, a decrease in movement completion time from 1.47 to 1.13, as well as a significant improvement in classifier accuracy from 77.5\% to 94.4\%, for users undergoing user training for a two week period. This study provided feedback through a virtual animated prosthesis.
Pen et. al \cite{Pen2017} provided a visual feedback of an arrow to be moved on a 2D plane. Pen et. al also tested the effect of stimulating the subjects brain with transcranial direct current stimulation (tDCS). The study concluded that tDCS together with user training provided significantly better results than user training alone. \cite{Pen2017}

The general challenge of user training is for the user to be able to consistently produce distinguishable muscle patterns. \cite{Powell2014} Therefore further research in user training could provide a vital leap towards more precise classification using current methods, as well as a faster user adaptation of myoelectric controlled prosthetics, but an effective way to properly provide feedback to the user have yet to be developed. 
%Further studies should for now concentrate on developing different feedback methods which should later be compared to determine an ideal method. 
This study will seek to develop a new method of feedback during user training, by providing the users with the confidence scores of a LDA classifier as used in \cite{Scheme2013} for a confidence-based rejection system control scheme, and the level of contraction of the performed movement as proposed in \cite{Englehart2003}. To the authors knowledge, these feedback parameters have not yet been provided in a combination in user training for myoelectric prosthesis users. %Initially the study will propose providing feedback via a bar chart each bar representing the estimated uncertainty of a classified movement. 

%The following section ought to be a overview section in the beginning of the method section:
%The study will consist of steps of data acquisition, user training and an online test. The user will undergo user training followed by a targets reaching Fitts' Law test. The study will have a test and control group, where the control group will be trained without visual feedback during user training. Ad hoc statistic comparisons will be made between online tests and maybe some other things we have not entirely decided upon yet. /cite(we decide)

The remainder of this paper will be structured in "x numbers" of sections. Section 2 will further describe the experimental setup, subject management and experimental protocol. Section 3 will describe the methods used to do something with the control off and the user training thing we do. Section 4 present the result, discussion and conclusion. 

