%paper introduction

\section{Introduction}			%skal måske fjernes afhængigt af hvordan vi sætter main op

Electromyography (EMG) is the recording and utilization of muscle generated electric potentials, widely used in control of functional prosthetics. The electric potentials recorded from the muscles are action potentials generated by the activation of a muscle contraction. The contraction force a muscle produce is related to the intensity of an EMG recording. The recorded EMG signals are processed through several steps of amplification, filtering and feature extraction before they are used for control in a myoelectric prosthesis. \cite{Cram2012, Fougner2012} For the actual control of the prosthetics several different methods for control schemes exist.

In recent years the research area of myoelectric prosthetics has been dominated by classification methods for control schemes. Classification attempt to classify similar patterns in EMG signals, between previously acquired data and new data \cite{Mendez2017} This approach for control have proved adequate when performing proportional and simultaneous movements in several degrees of freedom (DOF), but have lacked usability outside of clinical environments due to poor accuracy with change in limb position. This have resulted in a high rejection rate by users. 
%small section on how classification methods function?
However, recently regression methods have been introduced and have shown promising results of robust control while performing both proportional and simultaneous movements, while also having proved to effectively combat the effect of limb position changes \cite{Hwang2017, Hanhe2014}. This shows potential for regression based control schemes to be more reliable when used for performing daily life tasks, outside clinical environments.
%small section on how regression methods function?

However, in studies the use of regression based control have been directly based off of the output of a trained regression model. This approach have shown robust control across different limb positions but lacks accurate control when performing delicate/precise movements or when only performing movement in one DOF. [cite til 7.semester projekt]. So far many advancements have been made to improve the systems and control schemes to best recognize the performed movements by the users. However, far fewer studies have investigated the effect of improving the users ability to properly utilize the system. So far the significance of user training is not doubted, but an effective way to properly provide feedback to the user have yet to be found \cite{Fang2017}. Fang et. al \cite{Fang2017} evaluated the progress of the human learning ability in a pattern recognition based control scheme when providing classifier-feedback during user training.

Fang et. al \cite{Fang2017} used a clustering-feedback method to provide users with real-time visual feedback, to guide users to correctly perform movements based on the recorded EMG signals. The study showed great improvements for user training. 

(using label and clustering feedback methods) 




A possible way to increase accurate control is to include a probability/uncertainty estimation of movements. This can be achieved by combining regression and classification methods into one control scheme. (linear regression and LDA)

This study investigate the effect of implementing uncertainty estimation and probability of desired movements, when using linear regression as control scheme. 


\textbf{Keywords: plutypas, neon, shoeeater, puppyspray, snow-blow-show, knife, duck, duck.}