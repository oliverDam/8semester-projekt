%abstract

%\section{Abstract}					%prolly not needed

%clever abstract below
User training is an important part of overcoming the challenge of making prosthetic limbs more functional, as the systems depend on the user generating the same distinguishable muscle patterns when using the prosthesis. Different methods have been sought when teaching users to perform specific distinguishable movements. This study aims to investigate whether confidence feedback in training can improve the users ability to complete a performance test in an interactive GUI compared to a control group receiving single class feedback. The study will be designed to examine prosthetic control in lower arm prosthetics.
16 able-bodied subjects were recruited for the study, where they were instructed to perform six different hand movements in three intensities during data acquisition. This was then used to build a LDA based classifier the subjects were to use during the training, teaching them to perform specific movements at different intensities. Afterwards they were subjected to a Fitts' Law based test in a GUI to examine their ability to perform the trained movements. These steps were repeated for three sessions over three consecutive days.
The results showed no significant difference between the two groups and no improvement over the three sessions for either of the groups. There was no difference in the performance during training as well. A significant improvement in cluster density was found in the control group, where the third session resulted in more dense clusters both compared to the first session and to the test group.

\textit{\textbf{Keywords: surface electromyography, lower arm prosthetics, linear discriminant analysis, user training}}