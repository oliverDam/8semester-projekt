%abstract

%\section{Abstract}					%prolly not needed

%clever abstract below
The user rejection rate of myoelectric prosthetics is currently high, due to slow and inaccurate control. Previous studies have shown user training to be an important part of overcoming the challenge of making transradial upper limb prosthetics more accurate, as the control systems depend on the user generating the same distinguishable muscle patterns when using the prosthesis. Different methods have been sought when adapting users to perform specific distinguishable movements. This study aimed to investigate whether confidence score feedback from a LDA classifier during user training could improve user performance in a Fitts' Law test compared to a control group who only received label feedback. %The study will be designed to examine prosthetic control in lower arm prosthetics.
16 able-bodied subjects were recruited for the study; 8 subjects randomly assigned to each group. Each subject went through a three session experiment; one session per day over three consecutive days. During each session the subject received a 16 minutes user training and went subsequently through a Fitts' Law test to evaluate the performance.
%, where they were instructed to perform six different hand movements in three intensities during data acquisition. This was then used to build a LDA based classifier the subjects were to use during the training, teaching them to perform specific movements at different intensities. Afterwards they were subjected to a Fitts' Law based test in a GUI to examine their ability to perform the trained movements. These steps were repeated for three sessions over three consecutive days.
A significant improvement in cluster dispersion of EMG signals of separate movement was found in the control group, where the third session resulted in more dense clusters both when compared to the first session of the control group and third session of the test group. The results from the Fitts' Law test showed no significant difference between the two groups and no improvement over the three sessions for either of the groups. Overall, three sessions of user training with confidence score feedback showed to be an insufficient training period to observe a significant improvement within and between subject groups.

\textit{\textbf{Keywords: surface electromyography, lower arm prosthetics, linear discriminant analysis, user training, confidence scores}}