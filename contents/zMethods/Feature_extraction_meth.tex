

\section{Feature extraction} 

To obtain the best possible classification, four different features have been extracted from the time-domain (TD) using their respective extraction methods. The features are Mean Absolute Value, Zero crossings, Slope Sign Changes and Waveform length. These will from now on be referred to as MAV, ZC, SSC and WL respectively. This set of features is chosen based on their extensive popularity when used for classification purposes and effect in real-time myoelectric control schemes, along with showing fast computational speeds.\cite{Hudgins1993, Kamavuako2016, Scheme2010}



\textbf{MAV} is an estimate of the mean absolute value of the signal, $\bar{X_i}$. i is the chosen segment where N is the number of samples in i. MAV is derived from, 


\begin{flalign}
	\bar{X_i} = 1/N \sum_{k=1}^{X} |x_k| ~ ~ for ~ i = 1 ... I 
\end{flalign}

where $x_k$ is the $k^{th}$ sample in $i$ and $I$ being the total number segments. 


\textbf{ZC} Finder lige ud af om de skal bruges 


\textbf{SSC} Samme her 

\textbf{WL} provides information on the waveform complexity in each segment by calculating the cumulative length of the waveform over time. The length is defined as,  

\begin{flalign}
	l_0=\sum_{k-1}^{N}|\triangle x_k| 
\end{flalign}

where $\triangle x_k = x_k - x_{k-1}$, which is the difference in consecutive sample voltage values). 

representation af hver feature 


\cite{Hudgins1993}