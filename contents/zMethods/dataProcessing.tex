

\section{Data processing}

In this section the filtering and feature extraction process be described.

\subsection{Filtering of signal} \label{sec:prePros} 

As earlier mentioned to counter the limitation in sample frequency imposed by the MYB an anti-aliasing filter has been implemented before further processing of the signal.  


The data obtained with MYB is acquired with a sample frequency of 200 Hz this propose the error of aliasing due to the non sufficient sampling frequency of at least twice the highest frequency \cite{Pozzo2004}. Furthermore it is recommended in \cite{DeLuca2010} to filter the low-frequency spectrum to counter artifacts of movement. Presented obstacles in both the low-frequency and high-frequency level of the acquired signal has lead to an implementation of a second order bandpass filter from 10 Hz to 90 Hz.   





\subsection{Feature extraction}

To obtain the best possible classification, four different features have been extracted from the time-domain (TD) using their respective extraction methods. The features are Mean Absolute Value, Zero crossings, Slope Sign Changes and Waveform length. These will from now on be referred to as MAV, ZC, SSC and WL respectively. This set of features is chosen based on their extensive popularity when used for classification purposes and effect in real-time myoelectric control schemes, along with showing fast computational speeds.\cite{Hudgins1993, Kamavuako2016, Scheme2010}

